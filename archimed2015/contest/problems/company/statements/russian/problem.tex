\begin{problem}{Детский сад}{стандартный ввод}{стандартный вывод}{2 секунды}{64 мегабайта}

В младшей группе детского сада <<Телепузики>> всего $n$ детей. Каждый из них, как и любой четырехлетка, легко может начать плакать просто из-за того, что его одногруппники тоже плачут. Ну и что, что он не знает, в чем дело? Товарищи же не могут ошибаться. 

Воспитательница работает в детском саду уже много лет, и отлично разбирается в детском настроении. Ей достаточно посмотреть на ребенка, чтобы понять, насколько он сегодня плаксив: заплачет ли он сегодня сам из-за того, что компот невкусный, разрыдается ли из-за того, что Катя и Ваня уже плачут, а он еще нет, или же будет сосредоточенно играть c кубиками, не обращая внимание на слезы и сопли товарищей. 

Зная сегодняшнюю плаксивость каждого из детей, определите, будет ли сегодня рыдать вся группа одновременно, или обойдется без массовой истерики. 


\InputFile
В первой строке дано целое число $n$ ($ 1 \leqslant n \leqslant 1000$) "--- количество детей в группе.
В следующей строке через пробел перечислены $n$ чисел, причем $i$-е по счету число $q_i$ ($0 \leqslant q_i \leqslant n-1$ ) обозначает плаксивость $i$-го ребенка.  Число $q_i$ обозначает количество детей, которые должны заплакать, чтобы этот ребенок тоже заплакал. Если $q_i = 0$, значит, этот ребенок точно сегодня заплачет просто так, вне зависимости от своих товарищей. Считается, что ребенок не может начать плакать, если вокруг него не плачет нужное количество детей. Если ребенок начал плакать, то он уже не успокоится до вечера. 


\OutputFile
Выведите <<YES>>, если вся группа будет плакать одновременно, или <<NO>> иначе.   

\Examples

\begin{example}
\exmp{4
1 0 1 2
}{YES
}%
\exmp{3
1 1 1
}{NO
}%
\end{example}

\end{problem}
