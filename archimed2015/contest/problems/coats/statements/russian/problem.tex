\begin{problem}{Куртки}{стандартный ввод}{стандартный вывод}{2 секунды}{64 мегабайта}

Благотоворительные организации каждый год собирают деньги на теплую одежду бедным. У главного героя этой задачи есть целых две куртки, но это не мешает ему страдать. Одна из его курток~--- зимняя, а вторая~--- демисезонная (в ней приятно ходить осенью или весной). Куртки подобраны идеально: в зимней куртке комфортно при температуре в $x$ градусов или ниже, а в демисезонной -- при температуре выше $x$ градусов. В общем, жить бы ему и радоваться. Но откуда бы тогда появиться задаче?

Проблема нашего героя в том, что он, надевая сегодня не ту куртку, которую носил вчера, постоянно забывает переложить проездной, ключи и прочие полезные вещи в карман новой куртки. Немного подумав, он решил, что не совсем подходящая к сегодняшней температуре куртка~--- это не так плохо, как забытые вещи. Поэтому, если сегодня незначительно теплее, чем граничная температура, он все равно пойдет в зимней куртке, аналогично для демисезонной. Чуть более формально это звучит так: он меняет куртку с зимней на демисезонную, только если сегодня за окном есть хотя бы $x+d$ градусов, а с демисезонной на зимнюю~--- если за окном $x-d$ градусов или холоднее. Иногда ему, конечно, не очень комфортно на улице, но зато все вещи точно с собой.

По архиву прогноза погоды за последние $n$ дней определите, сколько дней главному герою этой задачи было некомфортно. Считается, что в первый день он вышел в той куртке, в которой в этот день комфортно.

\InputFile
В первой строчке даны два вещественных числа $x$ и $d$~--- граница температуры между куртками и отклонение температуры, которое герой задачи считает незначительным ($-89\leqslant x\leqslant 55$, $1 \leqslant d \leqslant 6$).

Во второй строчке дано целое число $n$, $1 \leqslant n \leqslant 10^5$~--- количество дней в архиве прогноза погоды.

В третьей строчке перечислены $n$ вещественных чисел $t_i$~--- температура в $i$-й день ($-89\leqslant t_i\leqslant 55$).

\OutputFile
Выведите одно число: количество дней, в которые герою задачи было некомфортно в той куртке, в которой он вышел в этот день.

\Examples

\begin{example}
\exmp{5 1
7
6 7 4 4 2 3 7
}{0
}%
\exmp{0 2
4
-1 1 -1 1
}{2
}%
\end{example}

\end{problem}
