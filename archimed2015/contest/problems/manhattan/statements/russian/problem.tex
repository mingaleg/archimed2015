\begin{problem}{Нью-Кэпитал}{стандартный ввод}{стандартный вывод}{2 секунды}{64 мегабайта}

В стране из предыдущей задачи много специалистов не только по защите детей, но и про проектированию городов. Поэтому, чтобы решить проблему пробок в перенаселенной столице раз и навсегда, было решено построить новую столицу и перенести все правительство туда. Сказано~--- сделано. 

Улицы в новой столице образуют правильную прямоугольную сетку, в которой все улицы пересекаются ровно через одну местную единицу длины. Вертикально идущие улицы называются улицами, а горизонтально идущие~--- аллеями. Всего в городе получилось $2000$ улиц и $2000$ аллей, поэтому, чтобы не придумывать много новых названий, их все просто пронумеровали. Улицы пронумеровали с запада на восток числами от $-1000$ до $999$, а аллеи~--- с юга на север, тоже числами от $-1000$ до $999$. Центром города считаются кварталы на пересечении улиц и аллей с номерами от $-100$ до $100$. 

Чтобы увеличить пропускную способность дорог в городе, было решено сделать все улицы и аллеи односторонними. По улицам с четными номерами разрешается ехать только с севера на юг, а по улицам с нечетными номерами~--- только с юга на север. Аналогично, по аллеям с четными номерами можно ехать только с востока на запад, а с нечетными~--- только с запада на восток. 

Сколько местных единиц длины придется проезжать мэру новой столицы каждый вечер, возвращась из мэрии города домой? И мэрия, и дом мэра находятся в центре города. Мэр едет домой кратчайшим путем, соблюдая, впрочем, правила дорожного движения.

\InputFile
В первой строке даны два числа $x_1$ и $y_1$~--- номер улицы и номер аллеи, на пересечении которых находится мэрия. В второй строке даны два числа $x_2$ и $y_2$~--- номер улицы и номер аллеи, на пересечении которых находится дом мэра. Все числа целые и не превосходят по модулю $100$. 

\OutputFile
Выведите одно число: длину кратчайшего пути от мэрии до дома мэра на автомобиле. 

\Examples

\begin{example}
\exmp{0 0
1 1
}{4
}%
\exmp{3 5
2 4
}{4
}%
\end{example}

\end{problem}
