\begin{problem}{Том и Джерри}{стандартный ввод}{стандартный вывод}{2 секунды}{64 мегабайта}

В этой задаче мы снова возвращаемся в младшую группу детского сада <<Телепузики>>. Чтобы окончательно успокоить детей, воспитательница решила включить им мультик про Тома и Джерри. Серия, которую сейчас смотрят дети, довольно-таки незамысловата~--- в ней Джерри развесил по потолку комнаты наковальни на веревках. Когда Том оказывается под очередной наковальней, Джерри перерезает веревку. Наковальня падает на Тома, Тому больно, всем остальным весело, дети смеются. В общем, вполне обычная серия.

А вам нужно по кадру из этой серии определить, упадет ли наковальня на Тома, если Джерри перережет веревку.


\InputFile
Вам дана ASCII-арт картинка, то есть картинка, нарисованная символами. На ней есть наковальня, привязанная веревкой к потолку, и кот Том. 
В первой строке даны числа $N$, $M$ ($4 \leqslant N \leqslant 100$, $1 \leqslant M \leqslant 100$).
Следующие $N$ строк состоят из $M$ символов каждая, и представляют собой саму картинку. 
Картинка устроена следующим образом:
\begin{itemize}
\item Первые $K_1$ строк в одной и той же позиции $X_1$ стоит символ <<|>>, в остальных~--- пробел. Это веревка.
\item Следующие $K_2$ строк в одних и тех же позициях с $X_2$ по $X_3$ стоит символ <<\#>>, в остальных~--- пробел. Это наковальня.
\item $2 \times X_1 = X_2 + X_3$, то есть наковальня подвешена за середину.
\item Следующие $K_3$ строк содержат только пробелы. Это пустота между наковальней и котом.
\item Следующие $N - K_1 - K_2 - K_3$ строк содержат произвольные символы. Любой символ, кроме пробела~--- часть кота. Существует хотя бы один непробельный символ.
\end{itemize}
Числа $K_1$, $K_2$, $K_3$ и $N - K_1 - K_2 - K_3$ ненулевые.

\OutputFile
Выведите <<YES>>, если при падении наковальня заденет Тома, в противном случае выведите <<NO>>.

\Examples

\begin{example}
\exmp{13~29
~~~~~~~~~~|~~~~~~~~~~~~~~~~~~
~~~~~~~~~~|~~~~~~~~~~~~~~~~~~
~~~~~~~~~~|~~~~~~~~~~~~~~~~~~
~~~~\#\#\#\#\#\#\#\#\#\#\#\#\#~~~~~~~~~~~~
~~~~\#\#\#\#\#\#\#\#\#\#\#\#\#~~~~~~~~~~~~
~~~~\#\#\#\#\#\#\#\#\#\#\#\#\#~~~~~~~~~~~~
~~~~~~~~~~~~~~~~~~~~~~~~~~~~~
~~~~~~~~~~~~~~~~~~~~~~~~~~~~~
~~~~~~~~~~~~/\textbackslash{}\_/\textbackslash{}~~~~~~~~~~~~
~~~~~~~~~~~~>\^{}.\^{}<.---.~~~~~~~
~~~~~~~~~~~\_'-`-'~~~~~)\textbackslash{}~~~~~
~~~~~~~~~~(6--\textbackslash{}~|--\textbackslash{}~(`.`-.~~
~~~~~~~~~~~~~~--'~~--'~~``-'~}{YES}%
\exmp{16~30
~~~~|~~~~~~~~~~~~~~~~~~~~~~~~~
~\#\#\#\#\#\#\#~~~~~~~~~~~~~~~~~~~~~~
~\#\#\#\#\#\#\#~~~~~~~~~~~~~~~~~~~~~~
~\#\#\#\#\#\#\#~~~~~~~~~~~~~~~~~~~~~~
~\#\#\#\#\#\#\#~~~~~~~~~~~~~~~~~~~~~~
~~~~~~~~~~~~~~~~~~~~~~~~~~~~~~
~~~~~~~~~~~~,~~~~~~~~~~~~~~~~~
~~~~~~~~~~~\textbackslash{})\textbackslash{}\_~~~~~~~~~~~~~~~
~~~~~~~~~~/~~~~'.~.---.\_~~~~~~
~~~~~~~~=P~\^{}~~~~~`~~~~~~'.~~~~
~~~~~~~~~`--.~~~~~~~/~~~~~\textbackslash{}~~~
~~~~~~~~~.-'(~~~~~~~\textbackslash{}~~~~~~|~~
~~~~~~~~(.-'~~~)-..\_\_>~~~,~;~~
~~~~~~~~(\_.--``~~~~(\_\_.-/~/~~~
~~~~~~~~~~~~~~~~.-.\_\_.-'.'~~~~
~~~~~~~~~~~~~~~~~'-...-'~~~~~~
}{NO}%
\end{example}

\end{problem}
