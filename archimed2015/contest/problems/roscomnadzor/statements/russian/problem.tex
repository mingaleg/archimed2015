\begin{problem}{Слова не пройдут}{стандартный ввод}{стандартный вывод}{2 секунды}{64 мегабайта}

Дети, как известно, все раньше и раньше начинают пользоваться интернетом. Теперь, когда у них возникают вопросы, они не бегут к родителям, а заходят в свою любимую поисковую систему и узнают ответ в интернете. Но вдруг они случайно найдут что-нибудь, что им знать пока рановато? Или, может быть, лучше не знать вообще никогда?

В одной стране эту проблему решили очень просто: был создан список запрещенных для использования в интернете слов. Ведь очевидно, что статья, в которой упоминается какое-нибудь нехорошее слово, не может научить ребенка ничему хорошему. Любой сайт, содержащий хотя бы одно слово из этого списка, теперь подлежит мгновенной блокировке. Невинный ребенок никогда не натолкнется на что-нибудь, про что ему еще рановато знать~--- такой статьи просто не найдется в интернете. Но злобные сайтовладельцы придумали способ обойти этот запрет: если вместо некоторых букв написать внешне похожие на них цифры, то прочитать этот текст все равно будет можно, а робот, проверяющий сайты на пригодность, не распознает в слове запрещенное~--- ведь формально его нет на сайте.

Ваша задача~--- помочь правительству этой страны защитить детей от вредной информации. Напишите программу, которая будет проверять, нет ли в данной строке запрещенного слова, учитывая возможное коварство сайтовладельцев. Известно, что сайтовладельцы иногда делают следующие замены: 
e $\Rightarrow$ 3, 
o $\Rightarrow$ 0,
i $\Rightarrow$ 1,
t $\Rightarrow$ 7,
a $\Rightarrow$ 4,
s $\Rightarrow$ 5.

\InputFile
В первой строке входных данных дана строка~--- текст с сайта. Во второй строке входных данных дана другая строка~--- запрещенное слово. Первая строка состоит из маленьких латинских букв и цифр, вторая строка состоит только из маленьких латинских букв. Длина каждой строки не превышает $100$.


\OutputFile
Выведите <<YES>>, если запрещенное слово встречается как подстрока в строке с сайта, и <<NO>> иначе. Возможно, в строке с сайта некоторые буквы изначально были заменены на цифры в соответствии с приведенными выше правилами. 

\Examples

\begin{example}
\exmp{inah0leinthegroundthereliv3dah0bb1t
hobbit
}{YES
}%
\exmp{whath4v3igotinmypocket
handses
}{NO
}%
\exmp{whath4veig0t1nmyp0ck37
knife
}{NO
}%
\exmp{wh4thav31go71nmyp0ck3t
stringofnothing
}{NO
}%
\end{example}

\end{problem}
