\section{Задача <<Мама, я диспетчер>>}


\begin{frame}
    \begin{center}
        \Huge Задача <<Мама, я диспетчер>>
    \end{center}
    ~\\~\\
    \begin{center}
        Автор задачи: Олег~Мингалёв\\
        Автор условия: Елизавета~Игнатьева\\
        Автор тестов: Олег~Мингалёв
    \end{center}
\end{frame}

\subsection{Постановка задачи}

\begin{frame}
    \frametitle{Постановка задачи}

    \begin{itemize}
        \item Самолёт подлетает к аэропорту в момент времени $t_i$, посадка занимает время $B$, самолёт, прилетевший во время посадки другого, отправляется делать дополнительные круги над городом, каждый занимает время $F$. За какое время сядут все самолёты?
    \end{itemize}
\end{frame}

\subsection{Решение}

\begin{frame}
    \frametitle{Реализация}

    \begin{itemize}
        \item Будем поддерживать значения массива $t_i$ --- в какой момент времени подлетает $i$-ый самолёт.
        \item Каждый раз на посадку будем отправлять первый прилетевший из оставшихся самолёт.
        \item Для каждого из оставшихся ещё не севших самолётов происходит одна из следующих ситуаций:
        \begin{itemize}
            \item Самолёт прилетает через время, не меньшее $B$ (времени посадки), то есть коллизии не происходит и нам ни о чём думать не нужно.
            \item Самолёт прилетает через время, меньшее $B$, то есть во время посадки другого самолёта. В таком случае мы должны отправить его выполнять дополнительный круг над городом, то есть заменить $t_i$ на $t_i + F$.
        \end{itemize}
    \end{itemize}
\end{frame}

\begin{frame}
    \frametitle{Реализация}

    \begin{itemize}
        \item Возможно, что самолётам придётся выполнять больше одного круга, пока происходт посадка другого воздушного судна. На самом деле, пока садится самолёт $j$, самолёту $i$ нужно выполнить C кругов. 
        $$ C = \left\lceil \frac{t_j + B - t_i}{F} \right\rceil$$.
        \item То есть $t_i$ мы заменяем не на $t_i + F$, а на $t_i + C\times{}F$.
    \end{itemize}
\end{frame}

\begin{frame}
    \begin{center}
        \Huge Вопросы?
    \end{center}
\end{frame}

