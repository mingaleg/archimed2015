\section{Задача <<Детский сад>>}


\begin{frame}
    \begin{center}
        \Huge Задача <<Детский сад>>
    \end{center}
    ~\\~\\
    \begin{center}
        Автор задачи: Валерия~Петрова\\
        Автор условия: Елизавета~Игнатьева\\
        Автор тестов: Елизавета~Игнатьева
    \end{center}
\end{frame}

\subsection{Постановка задачи}

\begin{frame}
    \frametitle{Постановка задачи}

    \begin{itemize}
        \item Есть $N$ детей, для $i$-того ребёнка определена величина его плаксивости $q_i$.
        \item Если в какой-то момент плачет $A$ детей, то все дети, у которых $q_i \le A$, тоже начинают плакать.
        \item Заплачут ли в итоге все дети?
    \end{itemize}
\end{frame}

\subsection{Решение}

\begin{frame}
    \frametitle{Сортировка}

    \begin{itemize}
        \item Если $q_i < q_j$, то $j$-ый ребёнок не может заплакать раньше $i$-ого.
        \item Значит дети начинают плакать в порядке увеличения их плаксивости.
        \item Отсортируем $q_i$.
        \item Будем поддерживать количество уже плачущих детей $A$ и идти в порядке увеличения $q_i$.
        \item Если текущее $q_i > A$, то и все следующие $q_j > A$, то есть больше никто не заплачет и ответ <<NO>>.
        \item Иначе увеличиваем $A$ на $1$ (ребёнок заплакал) и переходим к следующему ребёнку.
        \item Если дети кончились, значит все дети уже плачут и ответ <<YES>>.
    \end{itemize}
\end{frame}

\begin{frame}
    \begin{center}
        \Huge Вопросы?
    \end{center}
\end{frame}

