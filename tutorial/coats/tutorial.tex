\section{Задача <<Куртки>>}


\begin{frame}
    \begin{center}
        \Huge Задача <<Куртки>>
    \end{center}
    ~\\~\\
    \begin{center}
        Автор задачи: Елизавета~Игнатьева\\
        Автор условия: Елизавета~Игнатьева\\
        Автор тестов: Елизавета~Игнатьева
    \end{center}
\end{frame}

\subsection{Постановка задачи}

\begin{frame}
    \frametitle{Постановка задачи}

    \begin{itemize}
        \item Есть две куртки, первая рассчитана на температуру не выше $x$ градусов, вторая --- на температуру выше $x$ градусов.
        \item Куртка меняется с первой на вторую, если температура стала не ниже $x+d$ градусов.
        \item Куртка меняется со второй на первую, если температура стала не выше $x-d$ градусов.
        \item Дан список изменения температуры. Сколько раз будет надета куртка, не рассчитанная на актуальную температуру?
        \item Изначально надета куртка, рассчитанная на первую температуру.
    \end{itemize}
\end{frame}

\subsection{Решение}

\begin{frame}
    \frametitle{Реализация}

    \begin{itemize}
        \item Будем поддерживать переменную, в которой указана надетая в настоящий момент куртка. Инициализируем её в зависимости от первой температуры.
        \item По очереди для всех температур выполним следующие операции:
        \begin{itemize}
            \item Если надета первая куртка, а температура не ниже $x+d$, то наденем вторую куртку
            \item \textbf{Иначе}, если надета вторая куртка, а температура не  выше $x-d$, то наденем первую куртку\\~\\
            \item Если надета первая куртка, а температура выше $x$, то увеличим счётчик с ответом на один.
            \item Если надета вторая куртка, а температура не выше $x$, то увеличим счётчик с ответом на один.
        \end{itemize}
    \end{itemize}
\end{frame}

\begin{frame}
    \begin{center}
        \Huge Вопросы?
    \end{center}
\end{frame}

